%%%%%%%%%%%%%%%%%%%%%%%%%%%%%%%%%%%%%%%%%%%%%%%%%%%%%%%%%%%%%%%%%%%%%%%%

%%% LaTeX Template for ECAI Papers 
%%% Prepared by Ulle Endriss (version 0.9 of 2023-11-07)

%%% To be used with the ECAI class file provided by IOS Press:
%%% - https://vtex-soft.github.io/texsupport.iospress-ecai/
%%% You will need the latest versions of the following files:
%%% - ecai.cls (class file)
%%% - ecai.bst (bibliography style file)
%%% You also will need a bibliography file (such as mybibfile.bib).

%%%%%%%%%%%%%%%%%%%%%%%%%%%%%%%%%%%%%%%%%%%%%%%%%%%%%%%%%%%%%%%%%%%%%%%%

%%% Start your document with the \documentclass{} command.
%%% Use the first variant for the camera-ready paper.
%%% Use the second variant for submission (for double-blind reviewing).

\documentclass{ecai} 
%\documentclass[doubleblind]{ecai} 

%%%%%%%%%%%%%%%%%%%%%%%%%%%%%%%%%%%%%%%%%%%%%%%%%%%%%%%%%%%%%%%%%%%%%%%%

%%% Load any packages you require here. 

\usepackage{latexsym}
\usepackage{amssymb}
\usepackage{amsmath}
\usepackage{amsthm}
\usepackage{booktabs}
\usepackage{enumitem}
\usepackage{graphicx}
\usepackage{color}

%%%%%%%%%%%%%%%%%%%%%%%%%%%%%%%%%%%%%%%%%%%%%%%%%%%%%%%%%%%%%%%%%%%%%%%%

%%% Define any theorem-like environments you require here.

\newtheorem{theorem}{Theorem}
\newtheorem{lemma}[theorem]{Lemma}
\newtheorem{corollary}[theorem]{Corollary}
\newtheorem{proposition}[theorem]{Proposition}
\newtheorem{fact}[theorem]{Fact}
\newtheorem{definition}{Definition}

%%%%%%%%%%%%%%%%%%%%%%%%%%%%%%%%%%%%%%%%%%%%%%%%%%%%%%%%%%%%%%%%%%%%%%%%

%%% Define any new commands you require here.

\newcommand{\BibTeX}{\rm B\kern-.05em{\sc i\kern-.025em b}\kern-.08em\TeX}

%%%%%%%%%%%%%%%%%%%%%%%%%%%%%%%%%%%%%%%%%%%%%%%%%%%%%%%%%%%%%%%%%%%%%%%%

\begin{document}

%%%%%%%%%%%%%%%%%%%%%%%%%%%%%%%%%%%%%%%%%%%%%%%%%%%%%%%%%%%%%%%%%%%%%%%%

\begin{frontmatter}

%%% Use this command to specify your EasyChair submission number.
%%% In doubleblind mode, it will be printed on the first page.

\paperid{123} 

%%% Use this command to specify the title of your paper.

\title{Guidelines for Preparing a Paper for the \\
European Conference on Artificial Intelligence}

%%% Use this combinations of commands to specify all authors of your 
%%% paper. [[TODO: explain name commands]]
%%% Specifying your ORCID digital identifier is optional. 
%%% [[TODO: new commands for corresponding author and contribution notes]]

\author[A]{\fnms{First}~\snm{Author}\orcid{....-....-....-....}\thanks{Corresponding Author. Email: somename@university.edu.}\footnote{Equal contribution.}}
\author[B]{\fnms{Second}~\snm{Author}\orcid{....-....-....-....}\footnotemark}
\author[B,C]{\fnms{Third}~\snm{Author}\orcid{....-....-....-....}} 

\address[A]{Short Affiliation of First Author}
\address[B]{Short Affiliation of Second Author and Third Author}
\address[C]{Short Affiliation of Third Author}

\begin{abstract}
This document outlines the formatting instructions for submissions to 
the European Conference on Artificial Intelligence (ECAI). 
Use the source file as a template when writing your own paper. 
The abstract of your paper should be a short and accessible summary 
of your contribution, preferably no longer than 200 words. 
It should not include any references to the bibliography.
\end{abstract}

\end{frontmatter}

%%%%%%%%%%%%%%%%%%%%%%%%%%%%%%%%%%%%%%%%%%%%%%%%%%%%%%%%%%%%%%%%%%%%%%%%

\section{Introduction}

The European Conference of Artificial Intelligence (ECAI) is the leading 
discipline-wide conference on AI in Europe. Its history goes back all 
the way to the Summer Conference on Artificial Intelligence and 
Simulation of Behaviour held in July 1974 in Brighton. Nowadays, ECAI is 
organised annually under the auspices of the European Association for 
Artificial Intelligence (EurAI, see Figure~\ref{fig:eurai}).

\begin{figure}[h]
\centering
\includegraphics[width=2.5cm]{eurai}
\vspace*{-10pt}
\caption{Logo of the European Association for Artificial Intelligence.}
\vspace*{-15pt}
\label{fig:eurai}
\end{figure}

Your paper should be typeset in \LaTeX, using the ECAI class file 
provided (\texttt{ecai.cls}). Please do not modify the class file or any 
of the layout parameters.

For instructions on how to submit your work to ECAI and on matters such 
as page limits or referring to supplementary material, please consult 
the Call for Papers of the next edition of the conference. Keep in mind
that you must use the \texttt{doubleblind} option for submission.

%%%%%%%%%%%%%%%%%%%%%%%%%%%%%%%%%%%%%%%%%%%%%%%%%%%%%%%%%%%%%%%%%%%%%%%%

\section{Example for a section header}

Please use sentence case for section headers. You probably are familiar 
with \LaTeX, but here's a simple example for an equation: 
%
\begin{eqnarray}\label{eq:vcg}
p_i(\boldsymbol{\hat{v}}) & = &
\sum_{j \neq i} \hat{v}_j(f(\boldsymbol{\hat{v}}_{-i})) - 
\sum_{j \neq i} \hat{v}_j(f(\boldsymbol{\hat{v}})) 
\end{eqnarray}
%
Use the usual combination of \verb|\label{}| and \verb|\ref{}| to refer
to numbered equations, such as Equation~(\ref{eq:vcg}). Next, a theorem: 

\begin{theorem}[Fermat, 1637]\label{thm:fermat}
No triple $(a,b,c)$ of natural numbers satisfies the equation 
$a^n + b^n = c^n$ for any natural number $n > 2$.
\end{theorem}

\begin{proof}
A full proof can be found in the supplementary material.
\end{proof}

Table captions should be centred \emph{above} the table, while figure 
captions should be centred \emph{below} the figure.\footnote{Footnotes
should be placed \emph{after} punctuation marks (such as full stops).}
 
\begin{table}
\caption{Locations of selected conference editions.}
\vspace*{-15pt}
\centering
\begin{tabular}{ll@{\hspace{8mm}}ll} 
\toprule
AISB-1980 & Amsterdam & ECAI-1990 & Stockholm \\
ECAI-2000 & Berlin & ECAI-2010 & Lisbon \\
ECAI-2020 & \multicolumn{3}{l}{Santiago de Compostela (online)} \\
\bottomrule
\end{tabular}
\vspace*{-10pt}
\end{table}

%%%%%%%%%%%%%%%%%%%%%%%%%%%%%%%%%%%%%%%%%%%%%%%%%%%%%%%%%%%%%%%%%%%%%%%%

\section{Citations and references}

Make sure to include full bibliographic information for everything you 
cite, whether it is a book~\cite{pearl2009causality}, a journal article 
\cite{grosz1996collaborative,rumelhart1986learning,turing1950computing}, 
a conference paper \cite{kautz1992planning}, or a preprint 
\cite{perelman2002entropy}. The use of \BibTeX\ to prepare your list of 
references is highly recommended. 

%%%%%%%%%%%%%%%%%%%%%%%%%%%%%%%%%%%%%%%%%%%%%%%%%%%%%%%%%%%%%%%%%%%%%%%%

%%% Use this environment to include acknowledgements (optional).

\begin{ack}
You may wish to use this (optional) section to thank your reviewers or 
to acknowledge a funding agency. Use the \texttt{ack} environment to 
typeset your acknowledgements. This will ensure that the text is 
suppressed when you use the \texttt{doubleblind} option. 
Acknowledgements can be included on the extra page intended for references.
\end{ack}

%%%%%%%%%%%%%%%%%%%%%%%%%%%%%%%%%%%%%%%%%%%%%%%%%%%%%%%%%%%%%%%%%%%%%%%%

%%% Use this command to include your bibliography file.

\bibliography{mybibfile}

\end{document}
%%%%%%%%%%%%%%%%%%%%%%%%%%%%%%%%%%%%%%%%%%%%%%%%%%%%%%%%%%%%%%%%%%%%%%
